\nonstopmode{}
\documentclass[a4paper]{book}
\usepackage[times,inconsolata,hyper]{Rd}
\usepackage{makeidx}
\makeatletter\@ifl@t@r\fmtversion{2018/04/01}{}{\usepackage[utf8]{inputenc}}\makeatother
% \usepackage{graphicx} % @USE GRAPHICX@
\makeindex{}
\begin{document}
\chapter*{}
\begin{center}
{\textbf{\huge Package `JackiePackage'}}
\par\bigskip{\large \today}
\end{center}
\ifthenelse{\boolean{Rd@use@hyper}}{\hypersetup{pdftitle = {JackiePackage: Tools for Enhanced Linear Regression Analysis}}}{}
\ifthenelse{\boolean{Rd@use@hyper}}{\hypersetup{pdfauthor = {Jackie Xu}}}{}
\begin{description}
\raggedright{}
\item[Title]\AsIs{Tools for Enhanced Linear Regression Analysis}
\item[Version]\AsIs{0.0.0.9000}
\item[Description]\AsIs{This package offers a suite of functions designed to simplify and elucidate linear regression analysis for users across different statistical backgrounds. It features capabilities for fitting regression models, summarizing outcomes, predicting values with new data, and conducting variance analysis through ANOVA tables. The primary goal of the package is to streamline complex statistical analysis into more user-friendly operations while maintaining robust analytical rigor.}
\item[License]\AsIs{MIT}
\item[Encoding]\AsIs{UTF-8}
\item[Roxygen]\AsIs{list(markdown = TRUE)}
\item[RoxygenNote]\AsIs{7.3.2}
\item[Depends]\AsIs{R (>= 3.6.0)}
\item[Suggests]\AsIs{haven, knitr, rmarkdown, testthat (>= 3.0.0)}
\item[VignetteBuilder]\AsIs{knitr}
\item[Config/testthat/edition]\AsIs{3}
\item[NeedsCompilation]\AsIs{no}
\item[Author]\AsIs{Jackie Xu [aut, cre]}
\item[Maintainer]\AsIs{Jackie Xu }\email{Jixhaw@umich.edu}\AsIs{}
\end{description}
\Rdcontents{Contents}
\HeaderA{ANOVA}{ANOVA for Linear Regression Model}{ANOVA}
%
\begin{Description}
Performs an analysis of variance (ANOVA) for a linear regression model,
breaking down the variability explained by each predictor and the residuals.
\end{Description}
%
\begin{Usage}
\begin{verbatim}
ANOVA(model)
\end{verbatim}
\end{Usage}
%
\begin{Arguments}
\begin{ldescription}
\item[\code{model}] A list object returned by the \code{Fit} function, containing model details.
\end{ldescription}
\end{Arguments}
%
\begin{Value}
Prints an ANOVA table to the console, showing the contribution of each predictor.
\end{Value}
%
\begin{Examples}
\begin{ExampleCode}
# Example data
data <- data.frame(
  Age = runif(100, 20, 80),
  Sex = sample(0:1, 100, replace = TRUE),
  Depression = rnorm(100, 50, 10)
)
fit_result <- Fit(data, "Depression")
ANOVA(fit_result)
\end{ExampleCode}
\end{Examples}
\HeaderA{Fit}{Perform Linear Regression}{Fit}
%
\begin{Description}
This function performs linear regression using the ordinary least squares (OLS) method.
It calculates the coefficients for the predictor variables and the specified response variable.
The function also computes various model metrics such as R-squared, adjusted R-squared,
residuals, and fitted values.
\end{Description}
%
\begin{Usage}
\begin{verbatim}
Fit(data, response)
\end{verbatim}
\end{Usage}
%
\begin{Arguments}
\begin{ldescription}
\item[\code{data}] A data frame containing both the predictor variables and the response variable.

\item[\code{response}] A string specifying the column name of the response variable in the data frame.
\end{ldescription}
\end{Arguments}
%
\begin{Value}
A list containing:
\begin{itemize}

\item{} \code{coefficients}: A named vector of estimated coefficients, including the intercept.
\item{} \code{residuals}: A numeric vector of residuals (differences between observed and fitted values).
\item{} \code{fitted\_values}: A numeric vector of fitted values.
\item{} \code{sse}: Sum of squared errors (residual sum of squares).
\item{} \code{sst}: Total sum of squares (total variability in the response variable).
\item{} \code{r\_squared}: R-squared, a measure of how well the model explains the variability of the response variable.
\item{} \code{adj\_r\_squared}: Adjusted R-squared, accounts for the number of predictors in the model.
\item{} \code{X}: The design matrix used in the regression analysis.
\item{} \code{df\_total}: Total degrees of freedom (sample size minus one).
\item{} \code{df\_residuals}: Degrees of freedom for residuals (sample size minus number of predictors).

\end{itemize}

\end{Value}
%
\begin{Examples}
\begin{ExampleCode}
# Example data
data <- data.frame(
  Age = runif(100, 20, 80),
  Sex = sample(0:1, 100, replace = TRUE),
  Depression = rnorm(100, 50, 10)
)

# Perform linear regression
result <- Fit(data, "Depression")

# Print the coefficients
print(result$coefficients)
\end{ExampleCode}
\end{Examples}
\HeaderA{Predict}{Predict Response Using Linear Regression Model}{Predict}
%
\begin{Description}
This function generates predicted values for a response variable based on the coefficients
of a linear regression model and new predictor data. It ensures consistency by verifying
that the new data includes all the predictors used in the fitted model.
\end{Description}
%
\begin{Usage}
\begin{verbatim}
Predict(model, newdata)
\end{verbatim}
\end{Usage}
%
\begin{Arguments}
\begin{ldescription}
\item[\code{model}] A list object returned by the \code{Fit} function. This list should contain:
\begin{itemize}

\item{} \code{coefficients}: A named vector of model coefficients, including the intercept.
\item{} \code{X}: The design matrix from the fitted model (used to verify predictors).

\end{itemize}


\item[\code{newdata}] A data frame of new predictor values. The column names must match the predictor variables
used in the model, excluding the response variable. The order does not matter as long as the names match.
\end{ldescription}
\end{Arguments}
%
\begin{Value}
A numeric vector of predicted values for the response variable.
\end{Value}
%
\begin{Examples}
\begin{ExampleCode}
# Example data
data <- data.frame(
  Age = runif(100, 20, 80),
  Sex = sample(0:1, 100, replace = TRUE),
  Depression = rnorm(100, 50, 10)
)

# Fit the model
fit_result <- Fit(data, "Depression")

# New predictor data
new_data <- data.frame(
  Age = c(30, 40, 50),
  Sex = c(0, 1, 1) 
)

# Predict response values
predictions <- Predict(fit_result, new_data)
print(predictions)
\end{ExampleCode}
\end{Examples}
\HeaderA{Summary}{Summary of Linear Regression Model}{Summary}
%
\begin{Description}
This function provides a detailed summary for a linear regression model fitted using the \code{Fit} function.
The summary includes estimates of coefficients, their standard errors, t-values, p-values,
residual standard error, R-squared, adjusted R-squared, and F-statistic.
\end{Description}
%
\begin{Usage}
\begin{verbatim}
Summary(model)
\end{verbatim}
\end{Usage}
%
\begin{Arguments}
\begin{ldescription}
\item[\code{model}] A list object returned by the \code{Fit} function. See \code{Fit.R} for details.
\end{ldescription}
\end{Arguments}
%
\begin{Value}
Prints the summary of the linear regression model to the console, including:
\begin{itemize}

\item{} A table of coefficients with estimates, standard errors, t-values, and p-values.
\item{} Residual standard error and its degrees of freedom.
\item{} Multiple R-squared and adjusted R-squared.
\item{} F-statistic and its associated p-value.

\end{itemize}

\end{Value}
%
\begin{Examples}
\begin{ExampleCode}
# Example data
data <- data.frame(
  Age = runif(100, 20, 80),
  Sex = sample(0:1, 100, replace = TRUE),
  Depression = rnorm(100, 50, 10)
)

# Fit the model
result <- Fit(data, "Depression")

# Generate the summary
Summary(result)
\end{ExampleCode}
\end{Examples}
\printindex{}
\end{document}
